\documentclass[]{article}

%opening
\title{Report Biomarkers}
\author{Benedetta Corso, Letizia, Dalila Dattoli}

\begin{document}

\maketitle

\begin{abstract}
Abstract: introduce rationale of investing brain dopamine function and potential brain asymmetry in PD.

This study, which focuses on Parkinson's disease, aims to investigate the significance of left-right lateralization based on dopamine values using DAT-SPECT imaging data and their correlation with clinical symptoms in Parkinson’s disease patient (PD) and healthy controls (HC). DAT-SPECT (Dopamine Transporter Single Photon Emission Computed Tomography) is an imaging technique primarily used to visualize and measure the density of dopamine transporters in the brain.  Indeed, this technique is particularly useful in the diagnosis and evaluation of neurodegenerative diseases, such as Parkinson's disease and other forms of parkinsonism. The dataset, used to perform the analysis, consists of variables relating to demographic data, DAT-SPECT scores and finally the NP tests which are a combination of clinical, neurological, and imaging test since there is no specific test for the diagnosis of Parkinson’s disease.

\end{abstract}

\section{Introduction}

- Literature evidenced investigating brain dopamine lat in PD
- aims of the study, based on literature evidences, 
formulate and motivate study hypotheses

\section{Material and Methods}

- Dataset:
\newline
	- Provide a summary of the dataset (refers to data files and PPMI
	portal). Please note that only individual with baseline DAT imaging 
	where included.
\newline
	- The main target regions are Caudate and Putamen, quantified as 
	Signal Binding Ration (i.e. DATSCAN PUTAMEN L/R and 
	DATSCAN CAUDATE L/R)
\newline
- Research methods:
\newline
	- Provide a description of the methodology used to answer the research 
	questions. 
	\newline
	- Provide an extensive and motivated description of statistical analysis plan, including the metrics used to assess the biomarker 
	performances

\section{Results}

- A clear and concise description of the statistical results 
providing answers to the research questions
\newline
- A sensitivity analysis of the results to covariates, group  matching and data quality (e.g. missing data, data miss balance)

\section{Discussion}

- Direct answers to the research questions
\newline
- An overview of the limitations of the study
\newline
- A list of possible suggestions to improve the study in case 
someone will repeat it in future

\end{document}
